\chapter{Theoretical and experimental basics}
\label{theory}

\section{The Standard Model of Particle Physics}

The Standard Model of particle physics summarizes the current knowledge of fundamental particles and their interactions. The model applies to scales of 1 fm and below. Gravity, being the fourth fundamental force is not included as it is negligible for most phenomena at this scale.

The current view is that all matter is made out of three kinds of elementary particle being leptons quarks and mediators.
There are six leptons falling into three families according to their charge, electron number, muon number and tau number. 

Similiar to that there are six flavors of quarks separated by strangeness (S), charm (C), beauty (B), and truth (T). As the leptons the quarks fall into three generations.
For both kinds of particles the mass rises with the generations and each generation comes as a doublet. The first particle of each lepton doublet is uncharged and referred to as a neutrino while the second particle has charge \num{-1}.
For each quark doublet there is an element with fractional charge $-\frac{1}{3}$ and an element with fractional charge $\frac{2}{3}$.
To each of these particles exists an anti particle of opposite charge.

The third kind of particle included in the standard model is the mediator. Mediators are gauge bosons the exchange of which allows the particles to interact. There are four kinds of elemtary interactions of which the strong electromagnetic and weak interaction are included in the model. The fourth interaction is the gravitational force.
The gauge particles for the strong interaction are the gluons carrying colour charge, the electromagnetic mediator is the photon ($\gamma$) and the weak mediators are the $W^{\pm}$ and $Z$ bosons.
Tables \ref{lepton properties}, \ref{quark properties} and \ref{mediator properties} summarize the particles and their important properties.
\newpage


\begin{table}[h]
\centering
\caption{Lepton properties}
\label{lepton properties}
\begin{tabular}{l|l|S|S|S|S|}
\cline{2-6}
                                   & symbol        & \text{Charge}  & \text{\ensuremath{L_e}} & \text{\ensuremath{L_{\mu}}} & \text{\ensuremath{L_{\tau}}} \\ \cline{2-6} 
\multirow{2}{*}{First generation\{}  & $e$             & -1       & 1    & 0          & 0           \\ \cline{2-6} 
                                   & $\nu_e$        & 0        & 1    & 0          & 0           \\ \cline{2-6} 
\multirow{2}{*}{Second generation\{} & $\mu$           & -1       & 0    & 1          & 0           \\ \cline{2-6} 
                                   & $\nu_{\mu}$  & 0        & 0    & 1          & 0           \\ \cline{2-6} 
\multirow{2}{*}{Third generation\{}  & $\tau$          & -1       & 0    & 0          & 1           \\ \cline{2-6} 
                                   & $\nu_{\tau}$ & 0        & 0    & 0          & 1           \\ \cline{2-6} 
\end{tabular}
\end{table}

\begin{table}[h]
\centering
\caption{Quark properties}
\label{quark properties}
\begin{tabular}{l|l|S|S|l|l|l|l|l|l|}
\cline{2-10}
                                      & Symbol & \text{Charge Q}      & \text{mass}       & D  & U & S  & C & B  & T \\ \cline{2-10} 
\multirow{2}{*}{First generation \{}  & $d$    &\text{\ensuremath{-\frac{1}{3}}} & \SI{4.8}{MeV}    & -1 & 0 & 0  & 0 & 0  & 0 \\ \cline{2-10} 
                                      & $u$    & \text{\ensuremath{\frac{2}{3}}}  & \SI{2.3}{MeV}   & 0  & 1 & 0  & 0 & 0  & 0 \\ \cline{2-10} 
\multirow{2}{*}{Second generation \{} & $s$    & \text{\ensuremath{-\frac{1}{3}}} & \SI{95}{MeV}    & 0  & 0 & -1 & 0 & 0  & 0 \\ \cline{2-10} 
                                      & $c$    & \text{\ensuremath{\frac{2}{3}}}  & \SI{1275}{MeV}   & 0  & 0 & 0  & 1 & 0  & 0 \\ \cline{2-10} 
\multirow{2}{*}{Third generation \{}  & $b$    & \text{\ensuremath{-\frac{1}{3}}} & \SI{4180}{MeV}   & 0  & 0 & 0  & 0 & -1 & 0 \\ \cline{2-10} 
                                      & $t$    & \text{\ensuremath{\frac{2}{3}}}  & \SI{173210}{MeV} & 0  & 0 & 0  & 0 & 0  & 1 \\ \cline{2-10} 
\end{tabular}
\end{table}

\begin{table}[h]
\centering
\caption{Mediator properties}
\label{mediator properties}
\begin{tabular}{|l|l|l|l|l|}
\hline
Interaction     & Theory & Mediator        & Charge          & Coupling  \\ \hline
Strong          & QCD    & gluons (8)      & colour          & 1         \\ \hline
Electromagnetic & QED    & photon $\gamma$ & electric charge & $10^{-1}$ \\ \hline
Weak            & GSW    & $W^{\pm}, Z$    & weak isospin    & $20^{-6}$ \\ \hline
\end{tabular}
\end{table}

Given this the standard model of particle physics has been a very successful model for a very long time and still holds for most cases.
Nevertheless the model has some commonly known weaknesses and does not claim to be complete. For example the gravitational force is not included and in the standard model neutrinos are massless which would not allow the oscillations observed in neutrinos originating from the sun.
For further information check \cite{griffith08}, \cite{thomson13} and \cite{brock11}.

There should be a short sentence about the Higgs Boson but I am unsure what to wirte exactly ? 
\newpage





\section{The LHC and ATLAS}

The analysis for this thesis has been performed in the ATLAS collaboration. The ATLAS-Detector is one of the four main experiments at the LHC at Cern. Therefore this chapter provides a brief overview over the LHC and ATLAS focusing on the properties directly relevant for Particle Flow Analysis.

After a description of the ATLAS detector in general I will give some further information about the detector components directly relevant for the explanation of particle flow which are the tracking detector and the calorimeter. To make the advantages of particle flow understandable to the reader I will give a general description of these detectors and their properties irrespective of their detailed construction at ATLAS.

\subsection{The LHC}

The Large Hadron Collider ("LHC") is part of the facilities of the European Organization of Nuclear Research ("CERN") and was built to extend the frontiers of modern particle physics by delivering high luminosities and unprecedented high energies. The Collider is circular with a circumfence of \SI{26.659}{\km} and is located \SI{10}{\km} underground close to Geneva.\\
The LHC is designed to collide bunches of up to \num{d11} protons at a luminosity of \SI{d34}{\per\square\cm \per\s}. The beams are collided at four collision points representing the four main experiments at the LHC. Two of these are special-purpose detectors, namely LHCb and ALICE while the other two are general-purpose detectors.
The general-purpose experiments are CMS and ATLAS. The analysis in this thesis was performed on ATLAS data.
Figure \ref{fig:LHC} shows the LHC, the four detectors and its general location.
\begin{figure}[h]
  \centering
  \includegraphics[width=\figwidth]{CERN-all-experiments}
  \caption[Sketch of the LHC ring, the position of the experiments and
  the surrounding countryside.]{Sketch of the LHC ring, the position
    of the experiments and the surrounding countryside. The four big
    LHC experiments are indicated(ATLAS, CMS, LHC-B and ALICE)along with their injection lines(Point 1, 2, 4, 8)\cite{atlasfigures}}
  \label{fig:LHC}
\end{figure}


\subsection{The ATLAS Detector}

The ATLAS-Detector was developed to take advantage of the high energy available at the LHC enabling the observation of highly massive particles that lower energy accelerators were not able to create and that way bring new physics theory beyond the standard model of particle physics.
It is designed to be able to observe a maximized number of final stages being a so called general purpose-detector. This means that the detector should be able to identify all kinds of particles and still provides an accurate information about angle and momentum.
Figure \ref{fig:atlas} shows the outline of the ATLAS detector together with a rough scale in size. In the following explanations of its components are given from the inside to the outside.

\begin{figure}[h]
  \centering
  \includegraphics[width=\figwidth]{atlas-detector}
  \caption[Sketch of the ATLAS detector]{Sketch of the ATLAS detector \cite{atlasfigures}}
  \label{fig:atlas}
\end{figure}

Figure \ref{fig:atlas_sketch} shows the detector's components in a simplified way and allows to understand the order of the important detector parts in detail. The innermost part of the detector is a tracking detector in a field of a large solenoid coil to obtain charge and trajectory of charged particles.
The following two parts are the electromagnetic and hadronic calorimeter which together build the calorimeter part of the detector. One of them focuses on measuring the energy in electromagnetic showers while the other one is optimized for hadronic particle showers. The outermost part is a muon spectrometer because most of the particles that cross the calorimeters undetected are muons.


\begin{figure}[h]
  \centering
  \includegraphics[width=\figwidth]{atlas-abstract}
  \caption[Sketch of the transversal section of the ATLAS detector]{Scheme of the ATLAS-detector \cite{atlasfigures}}
  \label{fig:atlas_sketch}
\end{figure}

\subsection{Tracking detectors}

The go to method to measure the momenta of charged particles is based on tracking detectors, which detect and monitor charged particles leaving behind tracks of ionizations in any given medium allowing to reconstruct the particle's trajectories.
There are two main categories of tracking detectors. The first one uses a large gaseous volume in a strong electric field and is filled with an area of wires. The electric field makes the liberating electrons drift towards the wires where they cause a detectable signal.

The second type of detectors is based on semiconductor technology and is used in most modern detectors like ATLAS. Therefore I will describe this kind of tracking detectors in more detail.
If a charged particle traverses an appropriately doped semiconductor wafer for example a doped silicon wafer it creates electron hole pairs along its trail. If an electric field is applied to the semiconductor material the holes will drift in the direction of the electric field and can then be collected by p-n junctions.

Usually a tracking detector is structured into semiconductor strips or pixel with a magnitude of \SI{25}{\micro \metre}, which allows to precisely determine the position of the event. By relating a a set of events to a single particle and knowing the space and time of these one can then extrapolate the track from the triggered pixels. The common way of setting up such a tracking detector is an array of cylindrical semiconductor wafers in a magnetic field. Each wafer signal gives a rough estimation of the particle's location at the time and the curve given by the sum of signals allows to calculate its charge and momentum.

\begin{equation}
p \cdot cos \lambda = 0.3 BR
\end{equation}

%I need graphics here for sure

\subsubsection{Inner Detector}

The Tracking detectors of the ATLAS detector are called the Inner Detector, which consists of three sub-components, the Pixel detector (Pixel), the Semi-Conductor Tracker (SCT) and a Transition Radiation Tracker (TRT). Each of these sub-detectors is divided into the so called barrel part and two end-caps. The Inner Detector covers a region of $|\eta| < \num{2.5}$ which also limits the region in which Particle Flow can be used.

\subsubsection{Muon spectrometer}

The outermost part of the detector is formed by the muon tracking chambers, the so called muon spectrometer. The task of the spectrometer is to detect charged particles transversing the calorimeter undetected and to both trigger on them and to measure their energy. Due to these two functions the spectromeer is divided into two parts each dedicated to one of the tasks. The first part is the trigger chamber covering a range of $|\eta|<2.4$, followed by the high-precision chamber with a range of $|\eta|<2.7$. The main detector's support feet cause a further gap at about $\phi = \ang{300}$ and $\phi = \ang{270}$. 

The high-precision detector uses monitored drift tubes (MDTs) while the trigger chamber uses resistive-plate chambers (RPCs). The momentum calculation is then performed by the field of the torroid magnet.

%get more from Jan thesis


\subsection{Calorimeter}

In particle physics a calorimeter is a device to measure first and foremost the total energy of a particle. Most of the time additionally some positional information is taken.
The idea is that most particles loose all their momentum crossing the calorimeter-structure. Measuring the energy deposited this way gives a value for the particle's energy.
Usually a particle deposits its energy by initiating a particle shower,the energy of which is then collected and measured.
Calorimeters are distinguished by the main interaction of the particles one aims to detect. 
\subsubsection{Electromgnetic Calorimeter}

Electromagnetic calorimeters are designed to detect charged particles and measure their total energy. Usually these particles are electrons and photons. There are various methods to construct these detectors. An example would be the usage of inorganic scintillators. These scintillators should be optically transparent and have a short radiation length to contain the shower in a compact region. The detection can then be followed by photon detectors which measure the emitted light being proportional to the detected particle's energy. The energy resolution of these detectors is typically in the range of

\begin{equation}
\frac{\sigma_E}{E} \sim \frac{3 \% - 10 \%}{\sqrt{E/GeV}}.
\end{equation}

The electromagnetic calorimeter at ATLAS is a high-resolution and high-granularity liquid-argon sampling calorimeter witch using lead as absorber material. The calorimeter consists of two half-barrels which are only separated by a small gap at the interaction point. The endcaps at each side are segmented into two coaxial wheels to cover different polar angles.

\subsubsection{Hadronic calorimeters}

Hadronic calorimeters are used to obtain the energies of hadronic particle showers.
Due to the relatively large distance between interactions these calorimeters occupy a significantly large volume in the detector.

A common technique to construct these calorimeters is a sandwich-like structure of alternating layers of high density absorber material and active material. 
The absorbers are used to develop the particle showers which then hit the active material and deposit their energy there. That way hadronic calorimeters reach a resolution of about

\begin{equation}
\frac{\sigma_E}{E} \gtrsim \frac{50 \%}{\sqrt{E/GeV}}
\end{equation}

which is about one order of magnitude worse than for an electromagnetic calorimeter.

The hadronic calorimters at ATLAS are....
For more information see the ATLAS design report \cite{atlastdr}

