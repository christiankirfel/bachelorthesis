\chapter{Theoretical and experimental basics}
\label{theory}

\section{The Standard Model of Particle Physics}

The Standard Model of Particle Physics summarizes the current knowledge of fundamental particles and their interactions. The model holds at scales of 1 fm and below. Gravity, being the fourth fundamental force is not included because it is negligible for most phenomena at this scale.
The Standard Model describes the interactions of the fundamental spin  s = 1/2 fermions, the quarks and leptons. For the electromagnetic and the strong force the interaction is mediated by spin s=1 boson. For the electromagnetic interaction this boson is the massless photon and the strong interaction gets mediated by gluons.  The third and weak force is mediated by the W and Z bosons.~\cite{griffith08}





\section{The LHC and ATLAS}

The analysis for this thesis has been performed in the ATLAS collaboration. The ATLAS-Detector is one of the four big experiments at the LHC at Cern. Therefore this chapter gives a brief overview over the LHC and ATLAS focusing on the properties directly relevant for Particle Flow Analysis.

\subsection{The LHC}

The Large Hadron Collider ("LHC") at CERN was built to extend the frontiers of modern particle physics by delivering high luminosities and unprecedented high energies.\\

The LHC is designed to collide bunches of up to 10(11) protons

\subsection{The ATLAS Detector}


\begin{figure}[htbp]
  \centering
  \includegraphics[width=\figwidth]{CERN_all-experiments}
  \caption[Sketch of the LHC ring, the position of the experiments and
  the surrounding countryside.]{Sketch of the LHC ring, the position
    of the experiments and the surrounding countryside. The four big
    LHC experiments are indicated. The location of the injection lines
    and the SPS are also shown.}
  \label{fig:LHC}
\end{figure}

\section{The Particle Flow Algorithm}

Recently only either the Calorimeter or the tracker information was used to reconstruct Jets in ATLAS events. The Particle Flow Algorithm combines tracker and calorimeter information to achieve better resolution especially at lower energies. The main advantages of including the tracker information into reconstruction are listed here:

\begin{itemize}
\item better resolution at low pt
\item better angular resolution
\item better tru
\end{itemize}


\section{Calibration in Particle Flow and its difficulties at the time}