\chapter{Theoretical and experimental basics}
\label{theory}

\section{The Standard Model of Particle Physics}

The Standard Model of particle physics summarizes the current knowledge of fundamental particles and their interactions. The model applies to scales of 1 fm and below. Gravity, being the fourth fundamental force is not included as it is negligible for most phenomena at this scale.

The current view is that all matter is made out of three kinds of elementary particle being leptons quarks and mediators.
There are six leptons falling into three families according to their charge, electron number, muon number and tau number. 

Similiar to that there are six flavors of quarks separated by strangeness (S), charm (C), beauty (B), and truth (T). As the leptons the quarks fall into three generations.
For both kinds of particles the mass rises with the generations and each generation comes as a doublet. The first particle of each lepton doublet is uncharged and referred to as a neutrino while the second particle has charge \num{-1}.
For each quark doublet there is an element with fractional charge $-\frac{1}{3}$ and an element with fractional charge $\frac{2}{3}$.
To each of these particles exists an anti particle of opposite charge.

The third kind of particle included in the standard model is the mediator. Mediators are gauge bosons the exchange of which allows the particles to interact. There are four kinds of elemtary interactions of which the strong electromagnetic and weak interaction are included in the model. The fourth interaction is the gravitational force.
The gauge particles for the strong interaction are the gluons carrying colour charge, the electromagnetic mediator is the photon ($\gamma$) and the weak mediators are the $W^{\pm}$ and $Z$ bosons.
Tables \ref{lepton properties}, \ref{quark properties} and \ref{mediator properties} summarize the particles and their important properties.
\newpage


\begin{table}[h]
\centering
\renewcommand{\arraystretch}{1.2}
\begin{tabular}{l|l|S|S|S|S|}
\cline{2-6}
                                   & symbol        & \text{Charge}  & \text{\ensuremath{L_e}} & \text{\ensuremath{L_{\mu}}} & \text{\ensuremath{L_{\tau}}} \\ \cline{2-6} 
\multirow{2}{*}{First generation\{}  & $e$             & -1       & 1    & 0          & 0           \\ \cline{2-6} 
                                   & $\nu_e$        & 0        & 1    & 0          & 0           \\ \cline{2-6} 
\multirow{2}{*}{Second generation\{} & $\mu$           & -1       & 0    & 1          & 0           \\ \cline{2-6} 
                                   & $\nu_{\mu}$  & 0        & 0    & 1          & 0           \\ \cline{2-6} 
\multirow{2}{*}{Third generation\{}  & $\tau$          & -1       & 0    & 0          & 1           \\ \cline{2-6} 
                                   & $\nu_{\tau}$ & 0        & 0    & 0          & 1           \\ \cline{2-6} 
\end{tabular}
\caption{Lepton properties}
\label{lepton properties}
\end{table}

\begin{table}[h]
\centering
\renewcommand{\arraystretch}{1.2}
\begin{tabular}{l|l|S|S|l|l|l|l|l|l|}
\cline{2-10}
                                      & Symbol & \text{Charge Q}      & \text{mass [GeV]}       & D  & U & S  & C & B  & T \\ \cline{2-10} 
\multirow{2}{*}{First generation \{}  & $d$    &\text{\ensuremath{-\frac{1}{3}}} & 4.8   & -1 & 0 & 0  & 0 & 0  & 0 \\ \cline{2-10} 
                                      & $u$    & \text{\ensuremath{\frac{2}{3}}}  & 2.3   & 0  & 1 & 0  & 0 & 0  & 0 \\ \cline{2-10} 
\multirow{2}{*}{Second generation \{} & $s$    & \text{\ensuremath{-\frac{1}{3}}} & 95    & 0  & 0 & -1 & 0 & 0  & 0 \\ \cline{2-10} 
                                      & $c$    & \text{\ensuremath{\frac{2}{3}}}  & 1275   & 0  & 0 & 0  & 1 & 0  & 0 \\ \cline{2-10} 
\multirow{2}{*}{Third generation \{}  & $b$    & \text{\ensuremath{-\frac{1}{3}}} & 4180   & 0  & 0 & 0  & 0 & -1 & 0 \\ \cline{2-10} 
                                      & $t$    & \text{\ensuremath{\frac{2}{3}}}  & 173210 & 0  & 0 & 0  & 0 & 0  & 1 \\ \cline{2-10} 
\end{tabular}
\caption{Quark properties}
\label{quark properties}
\end{table}

\begin{table}[h]
\centering
\renewcommand{\arraystretch}{1.5}
\begin{tabular}{|l|l|l|l|l|}
\hline
Interaction     & Theory & Mediator        & Charge          & Coupling  \\ \hline
Strong          & QCD    & gluons (8)      & colour          & 1         \\ \hline
Electromagnetic & QED    & photon $\gamma$ & electric charge & $10^{-1}$ \\ \hline
Weak            & GSW    & $W^{\pm}, Z$    & weak isospin    & $20^{-6}$ \\ \hline
\end{tabular}
\caption{Mediator properties}
\label{mediator properties}
\end{table}

Given this the standard model of particle physics has been a very successful model for a very long time and still holds for most cases.
Nevertheless the model has some commonly known weaknesses and does not claim to be complete. For example the gravitational force is not included and in the standard model neutrinos are massless which would not allow the oscillations observed in neutrinos originating from the sun.
For further information check \cite{griffith08}, \cite{thomson13} and \cite{brock11}.

In 2012 the Higgs boson has been discovered at the Large Hadron Collider. It is a spin-0 scalar particle with a mass of $m_H$ $=$ $\SI{125}{\GeV}$ and it represents the mechanism which gives all particles their mass.

\newpage





\section{The LHC and ATLAS}

The analysis for this thesis has been performed in the ATLAS collaboration. The ATLAS-Detector is one of the four main experiments at the LHC at Cern. This section provides a brief overview of the LHC and ATLAS detector focusing on the aspects directly relevant for Particle Flow analysis.

In addition to a brief description of the ATLAS detector a more detailed explanation of tracker and calorimter is given since these components are directly relevant for the explanation of Particle Flow.

\subsection{The LHC}

The Large Hadron Collider ("LHC") located at the facilities of the European Organization of Nuclear Research ("CERN") close to Geneva was built to extend the frontiers of modern particle physics by delivering high luminosities and reaching unprecedented high energies. The hadronic collider has a circumference of about \SI{27}{\km} and is located on average \SI{100}{\meter} underground.\\
The LHC is designed to collide bunches of up to \num{d11} protons at a luminosity of \SI{d34}{\per\square\cm \per\s}. The beams collide at four points where the four main experiments of the LHC are located. Two of these are special-purpose detectors, namely LHCb and ALICE while the other two, ATLAS and CMS, are general-purpose detectors.
The analysis in this thesis was performed on ATLAS data.
Figure \ref{fig:LHC} shows the LHC, the four detectors and its general location.
\begin{figure}[h]
  \centering
  \includegraphics[width=\figwidth]{CERN-all-experiments}
  \caption[Sketch of the LHC ring, the position of the experiments and
  the surrounding countryside.]{Sketch of the LHC ring, the position
    of the experiments and the surrounding countryside. The four big
    LHC experiments are indicated(ATLAS, CMS, LHC-B and ALICE)along with their injection lines(Point 1, 2, 4, 8)\cite{atlasfigures}}
  \label{fig:LHC}
\end{figure}


\subsection{The ATLAS Detector}

The ATLAS detector was developed to study the physic processes in a broad energy range available at the LHC. This enables the observation of highly massive particles that lower energy accelerators were not able to create and that would bring new physics theory beyond the standard model of particle physics.
It was designed to cover the maximum number of final stages being a so called general purpose-detector.
Figure \ref{fig:atlas} is a sketch of the ATLAS detector together with a rough scale in size not only by the given dimensions on the top and left side but also by including two average sized humans close to the left muon chambers. In the following explanations of its components are given from the inside to the outside.

\begin{figure}[h]
  \centering
  \includegraphics[width=\figwidth]{atlas-detector}
  \caption[Sketch of the ATLAS detector]{Sketch of the ATLAS detector \cite{atlasfigures}}
  \label{fig:atlas}
\end{figure}

Figure \ref{fig:atlas_sketch} shows the detector's components in a simplified way and allows to understand the importance of the order of the detector's parts. The innermost part of the detector is a tracking detector surrounded by a solenoid that creates a magnetic field to bend the charged particles trajectory and measure their charge and momentum.
The following part of the detector is the calorimetry system. It consists of an inner electromagentic calorimeter and and outer hadronic calorimter. The outermost part is a muon spectrometer because most of the particles that cross the calorimeters undetected or do not deploy their complete energy are muons.

The detector system therefore allows to measure charge, momentum end energy of most particles.


\begin{figure}[h]
  \centering
  \includegraphics[width=\figwidth]{atlas-abstract}
  \caption[Sketch of the transversal section of the ATLAS detector]{Scheme of the ATLAS-detector \cite{atlasfigures}}
  \label{fig:atlas_sketch}
\end{figure}

\subsection{The ATLAS coordinate system}

The ATLAS coordinate system is defined by the beam direction with the $z$-axis pointing along the LHC's beam pipe. The corresponding transverse plane is defined by the $x$-axis pointing towards the ring's centre. The $y$-axis points upwards. The origin of the system is defined by the nominal point of interaction. The polar angle, $\theta$, is the angle between the $z$-axis and the $x$-$y$-plane and the azimuthal angle, $\phi$ is the angle between the $x$- and the $y$-axis.

The coordinates used in this thesis are usually the azimuthal angle, $\phi$, the pseudo-rapidity, $\eta$, and the transverse momentum, $p_T$. The pseudo-rapidity replaces the polar angle and is defined as

\begin{equation}
\eta = \frac{1}{2} ln\left[ tan\left(\frac{\theta}{2}\right)\right].
\end{equation}

The transverse momentum is defined by

\begin{equation}
p_T = \sqrt{p_x^2 + p_y^2}
\end{equation}
where $p_x$ and $p_y$ are the momenta along the corresponding axes. 

The angular variables are defined within

\begin{equation}
\eta \in [-\infty,\infty],\,
\phi \in [-\pi,\pi].
\end{equation}
\subsection{Tracking detectors}

To measure momentum, trajectory and charge of charged particles usually tracking detectors are used.

There are two main categories of tracking detectors following the same general principle, gaseous detectors and semiconductor detectors. If ionizing radiation passes any given medium it will create electron-hole pairs. These charge carriers can then by collected by an electric field. Depending on the detector the signal caused by the charge carriers can be related to the coordinate of ionization in space and time.

\begin{itemize}
\item Gaseous detectors: Gaseous detectors are based on the grater mobility of ions and electrons in the gas. The basis of the detector is usually a chamber filled with a proper gas. The gas is filled with an area of wires to which a strong electric field is applied. If gas atoms get ionized the charge carriers (electrons and ions) will drift to the wires and create a detectable signal. The wire gives a rough estimation of space which is normally improved by calculating the exact ionization location from the drift time.

\item Semiconductor detectors: Semiconductor detectors are as their name implies based on crystalline semiconductor material for example silicon and germanium. Their working principle is quite similar to that of gaseous detectors but the gas is exchanged by solid semiconducting material. In semiconducting material ionizing radiation will create electron-hole pairs instead of electron ion pairs that then can travel in a strong electric field to be detected. The big advantage of semiconductor detectors over gaseous detectors is that the energy required to create an electron-hole pair is about 10 times smaller than the energy needed to ionize gas atoms. These detectors are commonly structured into wafers or pixels that allow a determination of the space.
\end{itemize}

Usually a magnetic field surrounds tracking detectors to bend the track and that way be able to compute the particles momentum and charge based on the curvature.


In the ATLAS detector both the inner detector and the muon spectrometer are tracking detectors.

\subsubsection{Inner Detector}

The innermost part of the ATLAS detector is called the Inner Detector, which consists of three sub-components, the Pixel detector (Pixel), the Semi-Conductor Tracker (SCT) and a Transition Radiation Tracker (TRT). Each of these sub-detectors is divided into the so called barrel part and two end-caps. The Inner Detector covers a region of $|\eta| < \num{2.5}$ which also limits the region in which Particle Flow can be used as its highest efficiency.

\subsubsection{Muon spectrometer}

The second tracking detector of ATLAS is the muon spectrometer which is the outermost part of the detector. The task of the spectrometer is to detect charged particles transversing the calorimeter without being stopped or deploying their complete energy, and to do both trigger and tracking to measure their momentum. Due to these two tasks the spectrometer is divided into two parts. The first part is the trigger chamber covering a range of $|\eta|<2.4$, followed by the high-precision chamber with a range of $|\eta|<2.7$. The main detector's support feet cause a further gap at about $\phi = \ang{300}$ and $\phi = \ang{270}$. 

The high-precision detector uses monitored drift tubes (MDTs) with the exception of the innermost part of the innermost end-cap disk which utilizes Cathode Strip Chambers (CSCs). The trigger chamber uses resistive-plate chambers (RPCs) for the barrel parts and Thin-Gap Chambers (TGCs) for the end-caps. The momentum calculation is then performed by the field of the toroid magnet.



\subsection{Calorimeter}

In particle physics a calorimeter is a device to measure first and foremost the total energy of a particle. Most of the time additionally some positional information is taken.
The idea is that most particles loose all their momentum while crossing the calorimeter. Measuring the energy deposited this way gives a value for the particle's energy.
Usually a particle deposits its energy by initiating a particle shower,the energy of which is then collected and measured.
Calorimeters are distinguished by the main interaction of the particles one aims to detect. 
\subsubsection{Electromagnetic Calorimeter}

Electromagnetic calorimeters are designed to detect charged particles that primarily via the electromagnetic interaction and measure their total energy. Usually these particles are electrons and photons. There are various methods to construct these detectors. An example would be the usage of inorganic scintillators. These scintillators should be optically transparent and have a short radiation length to contain the shower in a compact region. The detection can then be followed by photon detectors with photo-multipliers which measure the emitted light being proportional to the detected particle's energy.

The electromagnetic calorimeter at ATLAS is a high-resolution and high-granularity liquid-argon sampling calorimeter witch using lead as absorber material. The calorimeter consists of two half-barrels which are only separated by a small gap at the interaction point. The endcaps at each side are segmented into two coaxial wheels to cover different polar angles.

\subsubsection{Hadronic calorimeters}

Hadronic calorimeters are used to obtain the energies of hadronic particles.
Due to the relatively large distance between interactions these calorimeters occupy a significantly large volume in the detector.

A common technique to construct these calorimeters is a sandwich-like structure of alternating layers of high density absorber material and active material. 
The absorbers are used to develop the particle showers which then hit the active material and deposit their energy there. The determination of the particle belonging to the deployed energy relies on tracker information as is sketched in figure \ref{fig:atlas_sketch}. Energy in the hadronic calorimeter without a track implies a neutral hadron, for example a neutron. A single track paired with a energy deposition means that the particle was a charged hadron like a proton and if many tracks belong to a deposition a jet has been the most likely origin of the energy deposition.


The hadronic calorimter system at ATLAS is divided into three calorimeter components. The first one is the scintillator-tile calorimter covering a region of $|\eta|<\num{1.7}$. The other two are the end-cap calorimters which use liquid argon (LAr) and cover the region of $\num{1.5}<|\eta|<\num{3.2}$.
The tile calorimeter itself is divided into a central barrel and two extended barrels (compare \ref{fig:atlas}).


For more information about the ATLAS detector see the ATLAS design report \cite{atlastdr}.

