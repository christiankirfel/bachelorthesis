%==============================================================================
\chapter{Introduction}
\label{sec:intro}
%==============================================================================

The Large Hadron Collider (LHC) is an accelerator at CERN near Geneva. One of its main experiments is the general-purpose detector ATLAS. This thesis describes the performance of a reconstruction algorithm named Particle Flow.


The Particle Flow jets have recently been stabilised as one of the official jet collections to be used in the ATLAS analysis. Particle Flow combines tracker and calorimeter information and previous studies\cite{pflow16} have demonstrated that it is a promising approach improving the angular resolution and transversal momentum resolution especially for lower transversal momentum. The aim of this thesis was to create a framework for the study of Particle Flow performance on 2016 \SI{13}{\TeV} data and to perform data-Monte Carlo comparison.

This thesis is structured as follows:

Chapter 2 gives a brief introduction to the standard model of particle physics, that describes the fundamental particles and their interactions. Furthermore the chapter includes a simple description of the ATLAS detector and a more detailed explanation of tracking detectors and calorimeters since they are important for Particle Flow.

The third chapter describes the Particle Flow algorithm in detail and also presents a brief overview of the Run 1 results as well as the changes that have been applied to the algorithm for Run 2.

Chapter four summarizes the analysis framework that has been developed during this thesis. It gives an explanation for all the important tools used in the framework and concludes by listing the tools that still have to be implemented or generated for Particle Flow.

Chapter 5 then finally presents the results derived from data/Monte Carlo comparison for 2016 data on $Z\rightarrow \mu \bar{\mu}$ events.




%%% Local Variables: 
%%% mode: latex
%%% TeX-master: "../mythesis"
%%% End: 
