%==============================================================================
\chapter{Introduction}
\label{sec:intro}
%==============================================================================




The Particle Flow algorithm is a promising approach of data analysis and has already been approved before on CMS as well as on ATLAS 2015 data. The combination of tracker  and calorimeter information allows to improve the resolution especially in lower energy regions and can therefore reduce the energy threshold for analysis. The aim of my thesis was to create an analysis framework allowing Particle Flow analysis on 2016 \SI{13}{\TeV} data and  to show the improvements the new algorithm can yield.
The second chapter of this thesis gives a brief introduction to the Standard Model of particle physics, the approved model used to describe most particle interactions, decays and crossections. Furthermore the chapter includes a description of the ATLAS detector and its components and in addition a more general explanation of the attributes of tracking detectors and calorimeters to explain why better results are expected from the combination of both.

In the third chapter I describe the Particle Flow algorithm in detail and also present a brief overview of the Run 1 results as well as of the changes that have been applied to the algorithm since its first application in ATLAS.

Chapter four then summarizes the actual analysis framework that has been developed during my thesis. I give an explanation of all the important tools used in the framework and conclude by listing the  difficulties and obstacles that still have to be overcome to complete a comprehensive analysis framework for Particle Flow.

Finally in chapter five I present the results derived from data/Monte Carlo comparison for 2016 data on $Z\rightarrow \mu \mu$ events.


%%% Local Variables: 
%%% mode: latex
%%% TeX-master: "../mythesis"
%%% End: 
