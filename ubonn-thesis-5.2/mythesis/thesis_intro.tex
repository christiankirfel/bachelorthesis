%==============================================================================
\chapter{Introduction}
\label{sec:intro}
%==============================================================================


The Particle Flow algorithm is a promising way of data analysis and has already been approved on CMS data aswell as on ATLAS 2015 data. The combination of tracker  and calorimeter information allows to improve the the resolution and overall results especially in lower energy regions and can therefore reduce the energy threshold for analysis. For my thesis I have worked on creating an analysis framework to allow Particle Flow analysis on 2016 \SI{13}{\TeV} data and showing the improvements the new algorithm can bring.
The first chapter of this thesis gives a brief introduction to the Standard Model of particle physics, the approved model that is used to describe most particle interactions, decays and crossections.
Before describing the actual particle flow algorithm and its results I give a description of the ATLAS detector and its components and furthermore a more general explanation of the attributes of tracking detectors and calorimeters to explain why one expects better results from the combination of both.
Last but not least the algorithm is explained in detail before the results of using it on 2016 data and Monte Carlo are given.

%%% Local Variables: 
%%% mode: latex
%%% TeX-master: "../mythesis"
%%% End: 
